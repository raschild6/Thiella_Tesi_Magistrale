%!TEX root = ../dissertation.tex
\chapter{Appendix A} \label{AppendixA}

\section{Knowledge Base Examples}\label{sec:KB_examples} 

\begin{footnotesize}
\textbf{Code in Atomese notation:} \\
Starting with the simplest arrangement of objects: there is a table and there is a screw, a bolt and a toolbox on it. Then, first the objects and their inheritances are defined, next the objects are differentiated, and finally, as they are all on top of the table, the \textit{clear} status is associated with each of them. \\
\end{footnotesize}

\begin{python}
	(InheritanceLink
		(ConceptNode "screw")
		(ConceptNode "object"))
	(InheritanceLink
		(ConceptNode "bolt")
		(ConceptNode "object"))
	(InheritanceLink
		(ConceptNode "toolbox")
		(ConceptNode "object"))
	(InheritanceLink
		(ConceptNode "table")
		(ConceptNode "fixed-object"))

	(NotLink (EqualLink 
		(ConceptNode "screw") (ConceptNode "bolt")))
	(NotLink (EqualLink 
		(ConceptNode "screw") (ConceptNode "toolbox")))
	(NotLink (EqualLink 
		(ConceptNode "screw") (ConceptNode "table")))
	(NotLink (EqualLink 
		(ConceptNode "bolt") (ConceptNode "toolbox")))
	(NotLink (EqualLink 
		(ConceptNode "bolt") (ConceptNode "table")))
	(NotLink (EqualLink 
		(ConceptNode "toolbox") (ConceptNode "table")))
	
	(EvaluationLink
		(PredicateNode "clear")
		(ConceptNode "screw"))
	(EvaluationLink
		(PredicateNode "clear")
		(ConceptNode "bolt"))
	(EvaluationLink
		(PredicateNode "clear")
		(ConceptNode "toolbox"))
	(EvaluationLink
		(PredicateNode "clear")
		(ConceptNode "table"))
\end{python}

Fixed objects do not lose their \textit{clear} state when an object is placed on top of them, as is the case of the other objects. However, this state is maintained as long as there is room to place a new object, otherwise it is removed and the fixed object will result completely busy. \\
Furthermore, notice how there are no states related to the fact that the objects are on top of the table. This is because the actions of \textit{Pickup} and \textit{Putdown} have been maintained. The table is considered as the main working plane, so these actions will be executed within the objects above it. \\
In general, these actions can be used on objects that are on top of fixed objects in the initial situation, so for example, \textit{Pickup} will not require to know what is under the object to pick up, it will just pick it up. \textit{Putdown}, on the other hand, is used to put down an object in a generic position, it is not important where. This action is rarely used. \\
However, all four actions are maintained and used to speed up the first actions and simplify the initial knowledge base.\\

To conclude, suppose now that the screw is on the toolbox and the bolt is in the hand. InheritanceLinks and NotLinks are the same as the code above and will always remain unchanged once created, within each execution of the algorithm, thus they are omitted below. The last part of the knowledge base will become as follows: \\

\begin{python}	
	(EvaluationLink
		(PredicateNode "clear")
		(ConceptNode "screw"))
	(EvaluationLink
		(PredicateNode "in-hand")
		(ConceptNode "bolt"))
	(EvaluationLink
		(PredicateNode "on")
		(ListLink
			(ConceptNode "screw")
			(ConceptNode "toolbox"))
	(EvaluationLink
		(PredicateNode "clear")
		(ConceptNode "table"))
\end{python}


\section{PDDL Problem and Domain}\label{sec:PDDL} 

\begin{footnotesize}
\textbf{Code in PDDL notation:} \\
Description of the problem as a definition of the domain, objects, initial state (INIT) and goal. \\
In this example: four blocks (A, B, C, D) are arranged independently on the table and the robot's hand is free. The goal is a stack formed by the four blocks placed in reverse alphabetical order (D on C on B on A).
\end{footnotesize}

\begin{python}
(define (problem BLOCKS-4-0)
	(:domain BLOCKS)
	(:objects D B A C - block)
	(:INIT (CLEAR C) (CLEAR A) (CLEAR B) (CLEAR D) 
		(ONTABLE C) (ONTABLE A) (ONTABLE B) 
		(ONTABLE D) (HANDEMPTY))
	(:goal (AND (ON D C) (ON C B) (ON B A)))
)
\end{python}

\bigskip

\begin{footnotesize}
\textbf{Code in PDDL notation:} \\
Description of the domain, concerning the problem above, as all possible actions and \enquote{predicates} available. \\
Predicates are the \enquote{states} in which one or more objects can be. Actions, takeable by the robot, are described as input parameters and preconditions for performing that action and effects once the action is completed. 
\end{footnotesize}

\begin{python}
(define (domain BLOCKS)
  (:requirements :strips :typing)
  (:types block)
  (:predicates (on ?x - block ?y - block)
	       (ontable ?x - block)
	       (clear ?x - block)
	       (handempty)
	       (holding ?x - block)
	       )

  (:action pick-up
	     :parameters (?x - block)
	     :precondition (and (clear ?x) (ontable ?x) 
  			     (handempty))
	     :effect
	     (and (not (ontable ?x))
		   (not (clear ?x))
		   (not (handempty))
		   (holding ?x)))

  (:action put-down
	     :parameters (?x - block)
	     :precondition (holding ?x)
	     :effect
	     (and (not (holding ?x))
		   (clear ?x)
		   (handempty)
		   (ontable ?x)))
  (:action stack
	     :parameters (?x - block ?y - block)
	     :precondition (and (holding ?x) (clear ?y))
	     :effect
	     (and (not (holding ?x))
		   (not (clear ?y))
		   (clear ?x)
		   (handempty)
		   (on ?x ?y)))
  (:action unstack
	     :parameters (?x - block ?y - block)
	     :precondition (and (on ?x ?y) (clear ?x) 
			     (handempty))
	     :effect
	     (and (holding ?x)
		   (clear ?y)
		   (not (clear ?x))
		   (not (handempty))
		   (not (on ?x ?y)))))
\end{python}


\section{Action Rules}\label{sec:action_atomese}

\begin{footnotesize}
\textbf{Code in Atomese notation:} \\
\textit{Putdown} action rule. This rule works exactly in the same way as \textit{Pickup}. The only differences are the preconditions and effects, which are reversed, and the absence of the \enquote{stacked} state atom, which is no longer required. 
\end{footnotesize}

\begin{python}
(define (putdown-action . args)
  (define precond (car (cdr args)))
  (define eff (car args))
  (cog-extract-recursive! precond)
  eff
)

(define (putdown args)
  (let
    ((rule
      (if (equal? args `())
        (QueryLink
          (VariableList
            (TypedVariableLink
              (VariableNode "?ob") 
              (TypeNode "ConceptNode"))
          ) ; parameters
          (PresentLink
            (InheritanceLink
              (VariableNode "?ob")
              (ConceptNode "object"))
            (EvaluationLink
              (PredicateNode "in_hand")
              (VariableNode "?ob"))
          )
          (ExecutionOutputLink
            (GroundedSchemaNode "scm: putdown-action")
            (ListLink
              ; effect
              (EvaluationLink
                (PredicateNode "clear")
                (VariableNode "?ob")
              )
              ; precondition
              (EvaluationLink
                (PredicateNode "in_hand")
                (VariableNode "?ob")
              )
            )
          )
        )
        (QueryLink
          (VariableList
            (TypedVariableLink
              (VariableNode "?ob") 
              (TypeNode "ConceptNode"))
          ) ; parameters
          (AndLink
            (PresentLink
              (InheritanceLink
                (VariableNode "?ob")
                (ConceptNode "object"))
              (EvaluationLink
                (PredicateNode "in_hand")
                (VariableNode "?ob"))
            )
            (EqualLink
              (VariableNode "?ob")
              args
            )
          )
          (ExecutionOutputLink
            (GroundedSchemaNode "scm: putdown-action")
            (ListLink
              ; effect
              (EvaluationLink
                (PredicateNode "clear")
                (VariableNode "?ob")
              )
              ; precondition
              (EvaluationLink
                (PredicateNode "in_hand")
                (VariableNode "?ob")
              )
            )
          )
        )
      )
    ))
    rule
  )
)
\end{python}

\bigskip

\begin{footnotesize}
\textbf{Code in Atomese notation:} \\
\textit{Stack} action rule. This rule requires a more complex pattern than the first two, however, its structure and operation behave in the same way as the previous ones. \\
Firstly, two arguments are used to define which objects will form the stack.
Also in the pattern, an OrLink type atom differentiates the case where a fixed object is coinvolved. 
Finally, the Scheme function \textit{stack-action} must first extract the object above and the object below and then it can use them to create the effect atoms and remove the state atoms that are no longer valid.
\end{footnotesize}

\begin{python}
(define (stack-action . args)
  (define precond (car (cdr args)))
  (define eff (car args))
  (define over_obj 
    (cog-outgoing-atom (cog-outgoing-atom eff 1) 0))
  (define under_obj 
    (cog-outgoing-atom (cog-outgoing-atom eff 1) 1))

  (cog-extract-recursive! precond)
  (cog-extract! (Predicate "in_hand"))

  (cog-extract! (Evaluation (Predicate "clear") under_obj))
  (cog-extract! (Predicate "clear"))

  (EvaluationLink (PredicateNode "clear") over_obj)
  (EvaluationLink (PredicateNode "stacked") over_obj)

  eff
)

(define (stack ob underob)
  (let
    ((rule
      (if (equal? ob `())
        (QueryLink
          (VariableList
            (TypedVariableLink 
              (VariableNode "?ob") 
              (TypeNode "ConceptNode"))
            (TypedVariableLink 
              (VariableNode "?underob") 
              (TypeNode "ConceptNode"))
          ) ; parameters
          (AndLink
            (PresentLink
              (NotLink
                (EqualLink 
                  (VariableNode "?ob") 
                  (VariableNode "?underob")))
              (InheritanceLink
                (VariableNode "?ob")
                (ConceptNode "object"))
              (EvaluationLink
                (PredicateNode "in_hand")
                (VariableNode "?ob"))
            )
            (OrLink
              (AndLink
                (InheritanceLink
                  (VariableNode "?underob")
                  (ConceptNode "object"))
                (EvaluationLink
                  (PredicateNode "clear")
                  (VariableNode "?underob"))
              )
              (AndLink
                (InheritanceLink
                  (VariableNode "?underob")
                  (ConceptNode "fixed-object"))
              )
            )
          )
          (ExecutionOutputLink
            (GroundedSchemaNode "scm: stack-action")
            (ListLink
              ; effect
              (EvaluationLink
                (PredicateNode "on")
                (ListLink
                  (VariableNode "?ob")
                  (VariableNode "?underob")
                )
              )
              ; precondition
              (EvaluationLink
                (PredicateNode "in_hand")
                (VariableNode "?ob"))
            )
          )
        )
        (QueryLink
          (VariableList
            (TypedVariableLink 
              (VariableNode "?ob") 
              (TypeNode "ConceptNode"))
            (TypedVariableLink 
              (VariableNode "?underob") 
              (TypeNode "ConceptNode"))
          ) ; parameters
          (AndLink
            (PresentLink
              (NotLink
                (EqualLink 
                  (VariableNode "?ob") 
                  (VariableNode "?underob")))
              (InheritanceLink
                (VariableNode "?ob")
                (ConceptNode "object"))
              (EvaluationLink
                (PredicateNode "in_hand")
                (VariableNode "?ob"))
            )
            (EqualLink
              (VariableNode "?ob")
              ob
            )
            (EqualLink
              (VariableNode "?underob")
              underob
            )
            (OrLink
              (AndLink
                (InheritanceLink
                  (VariableNode "?underob")
                  (ConceptNode "object"))
                (EvaluationLink
                  (PredicateNode "clear")
                  (VariableNode "?underob"))
              )
              (AndLink
                (InheritanceLink
                  (VariableNode "?underob")
                  (ConceptNode "fixed-object"))
              )
            )
          )
          (ExecutionOutputLink
            (GroundedSchemaNode "scm: stack-action")
            (ListLink
              ; effect
              (EvaluationLink
                (PredicateNode "on")
                (ListLink
                  (VariableNode "?ob")
                  (VariableNode "?underob")
                )
              )
              ; precondition
              (EvaluationLink
                (PredicateNode "in_hand")
                (VariableNode "?ob"))
             )
          )
        )
      )
    ))
    rule
  )
)
\end{python}

\bigskip

\begin{footnotesize}
\textbf{Code in Atomese notation:} \\
\textit{Unstack} action rule. Same considerations made for the rule of \textit{Putdown}: this rule behaves like \textit{Stack} with inverted preconditions and effects. While \textit{Stack} creates the \enquote{stacked} state atom, \textit{Unstack} removes it.
\end{footnotesize}

\begin{python}
(define (unstack-action . args)
  (define precond (car (cdr args)))
  (define eff (car args))
  (define over_obj 
    (cog-outgoing-atom (cog-outgoing-atom precond 1) 0))
  (define under_obj 
    (cog-outgoing-atom (cog-outgoing-atom precond 1) 1))

  (cog-extract-recursive! precond)
  (cog-extract! (Predicate "on"))
  (cog-extract! (List over_obj under_obj))

  (cog-extract! (Evaluation (Predicate "clear") over_obj))
  (cog-extract! (Predicate "clear"))
  (cog-extract! (Evaluation (Predicate "stacked") over_obj))
  (cog-extract! (Predicate "stacked"))

  (EvaluationLink (PredicateNode "clear") under_obj)

  eff
)

(define (unstack ob underob)
  (let
    ((rule
      (if (equal? ob `())
        (QueryLink
          (VariableList
            (TypedVariableLink 
              (VariableNode "?ob") 
              (TypeNode "ConceptNode"))
            (TypedVariableLink 
              (VariableNode "?underob") 
              (TypeNode "ConceptNode"))
          ) ; parameters
          (AndLink
            (PresentLink
              (NotLink
                (EqualLink 
                  (VariableNode "?ob") 
                  (VariableNode "?underob")))
              (InheritanceLink
                (VariableNode "?ob")
                (ConceptNode "object"))
              (EvaluationLink
                (PredicateNode "clear")
                (VariableNode "?ob"))
              (EvaluationLink
                (PredicateNode "on")
                (ListLink
                  (VariableNode "?ob")
                  (VariableNode "?underob")
                )
              )
            )
            (OrLink
              (InheritanceLink
                (VariableNode "?underob")
                (ConceptNode "object"))
              (InheritanceLink
                (VariableNode "?underob")
                (ConceptNode "fixed-object"))
            )
          )
          (ExecutionOutputLink
            (GroundedSchemaNode "scm: unstack-action")
            (ListLink
              ; effect
              (EvaluationLink
                (PredicateNode "in_hand")
                (VariableNode "?ob"))
              ; precondition
              (EvaluationLink
                (PredicateNode "on")
                (ListLink
                  (VariableNode "?ob")
                  (VariableNode "?underob")
                )
              )
            )
          )
        )
        (QueryLink
          (VariableList
            (TypedVariableLink 
              (VariableNode "?ob") 
              (TypeNode "ConceptNode"))
            (TypedVariableLink 
              (VariableNode "?underob") 
              (TypeNode "ConceptNode"))
          ) ; parameters
          (AndLink
            (PresentLink
              (NotLink
                (EqualLink 
                  (VariableNode "?ob") 
                  (VariableNode "?underob")))
              (InheritanceLink
                (VariableNode "?ob")
                (ConceptNode "object"))
              (EvaluationLink
                (PredicateNode "clear")
                (VariableNode "?ob"))
              (EvaluationLink
                (PredicateNode "on")
                (ListLink
                  (VariableNode "?ob")
                  (VariableNode "?underob")
                )
              )
            )
            (OrLink
              (InheritanceLink
                (VariableNode "?underob")
                (ConceptNode "object"))
              (InheritanceLink
                (VariableNode "?underob")
                (ConceptNode "fixed-object"))
            )
            (EqualLink
              (VariableNode "?ob")
              ob
            )
            (EqualLink
              (VariableNode "?underob")
              underob
            )
          )
          (ExecutionOutputLink
            (GroundedSchemaNode "scm: unstack-action")
            (ListLink
              ; effect
              (EvaluationLink
                (PredicateNode "in_hand")
                (VariableNode "?ob"))
              ; precondition
              (EvaluationLink
                (PredicateNode "on")
                (ListLink
                  (VariableNode "?ob")
                  (VariableNode "?underob")
                )
              )
            )
          )
        )
      )
    ))
    rule
  )
)
\end{python}
 
\bigskip

\begin{footnotesize}
\textbf{Code in Atomese notation:} \\
This is an additional rule of utility: given an object as an argument, it returns the object immediately under it, if it exists. \\
It is used by the BFS-Based algorithm \ref{sec:bfs_search} when calling \textit{Unstack} action.
\end{footnotesize}

\begin{python}
(define (find_underob args)
  (QueryLink
    (VariableList
      (TypedVariableLink 
        (VariableNode "?underob") 
        (TypeNode "ConceptNode"))
    ) ; parameters
    (AndLink
      (PresentLink
        (NotLink
          (EqualLink args 
          (VariableNode "?underob")))
        (EvaluationLink
          (PredicateNode "on")
            (ListLink
              args
              (VariableNode "?underob")
            )
        )
      )
      (OrLink
        (InheritanceLink
          (VariableNode "?underob")
          (ConceptNode "object"))
        (InheritanceLink
          (VariableNode "?underob")
          (ConceptNode "fixed-object"))
      )
    )
    (VariableNode "?underob")
  )
)
\end{python}


\section{NLP Rules}\label{sec:NLP} 

\begin{footnotesize}
\textbf{Code in Atomese notation:} \\
This rule infers all objects that are subject to a state but are independent of other objects. That is, the \textit{clear} and \textit{in-hand} state atoms.
\end{footnotesize}

\begin{python}
(define find_goal
  (QueryLink
    (VariableList
      (TypedVariableLink 
        (VariableNode "?status_id") 
        (TypeNode "PredicateNode"))
      (TypedVariableLink 
        (VariableNode "?status") 
        (TypeNode "PredicateNode"))
      (TypedVariableLink 
        (VariableNode "?ob_id") 
        (TypeNode "ConceptNode"))
      (TypedVariableLink 
        (VariableNode "?ob") 
        (TypeNode "ConceptNode"))
  
    )
    (PresentLink
      (EvaluationLink
        (VariableNode "?status_id")
        (ListLink
          (VariableNode "?ob_id")
        )
      )
      (InheritanceLink
        (VariableNode "?ob_id")
        (VariableNode "?ob")
      )
      (ImplicationLink
        (VariableNode "?status_id")
        (VariableNode "?status")
      )
    )
    (EvaluationLink
      (VariableNode "?status")
      (VariableNode "?ob")
    )
  )
)
\end{python}

\bigskip

\begin{footnotesize}
\textbf{Code in Atomese notation:} \\
This rule infers all objects subject to a stack. That is, the \textit{on} state atoms. \\
However, the various EvaluationLink type atoms (lines 78-84) that you get as results should already exist once Relex2Logic is run. \\
Unfortunately, it has errors which, instead of creating a single EvaluationLink that joins multiple ConceptNodes in the ListLink, create multiple EvaluationLinks with the same PredicateNode but separate for each object ( thus, the ListLinks contain only one ConceptNode each). \\
Since Relex2Logic is no longer maintained, this rule fixes this kind of error.
\end{footnotesize}

\begin{python}
(define fix
  (QueryLink
    (VariableList
      (TypedVariableLink 
        (VariableNode "?d1") 
        (TypeNode "DefinedLinguisticRelationshipNode"))
      (TypedVariableLink 
        (VariableNode "?d2") 
        (TypeNode "DefinedLinguisticRelationshipNode"))
      (TypedVariableLink 
        (VariableNode "?ob_w") 
        (TypeNode "WordInstanceNode"))
      (TypedVariableLink 
        (VariableNode "?underob_w") 
        (TypeNode "WordInstanceNode"))
      (TypedVariableLink 
        (VariableNode "?pred_w") 
        (TypeNode "WordInstanceNode"))
      (TypedVariableLink 
        (VariableNode "?ob") 
        (TypeNode "ConceptNode"))
      (TypedVariableLink 
        (VariableNode "?underob") 
        (TypeNode "ConceptNode"))
      (TypedVariableLink 
        (VariableNode "?pred") 
        (TypeNode "PredicateNode"))
      (TypedVariableLink 
        (VariableNode "?ob_id") 
        (TypeNode "ConceptNode"))
      (TypedVariableLink 
        (VariableNode "?underob_id") 
        (TypeNode "ConceptNode"))
      (TypedVariableLink 
        (VariableNode "?pred_id") 
        (TypeNode "PredicateNode"))
    )
    (PresentLink
      (EvaluationLink
        (VariableNode "?d1")
        (ListLink
          (VariableNode "?pred_w")
          (VariableNode "?underob_w")
        )
      )
      (EvaluationLink
        (VariableNode "?d2")
        (ListLink
          (VariableNode "?ob_w")
          (VariableNode "?pred_w")
        )
      )
      (InheritanceLink
        (VariableNode "?ob_id")
        (VariableNode "?ob")
      )
      (InheritanceLink
        (VariableNode "?underob_id")
        (VariableNode "?underob")
      )
      (ImplicationLink
        (VariableNode "?pred_id")
        (VariableNode "?pred")
      )
      (ReferenceLink
        (VariableNode "?ob_id")
        (VariableNode "?ob_w")
      )
      (ReferenceLink
        (VariableNode "?underob_id")
        (VariableNode "?underob_w")
      )
      (ReferenceLink
        (VariableNode "?pred_id")
        (VariableNode "?pred_w")
      )
    )
    (EvaluationLink
      (VariableNode "?pred")
      (ListLink
        (VariableNode "?ob")
        (VariableNode "?underob")
      )
    )
  )
)
\end{python}